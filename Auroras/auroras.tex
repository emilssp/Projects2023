
\documentclass[final,5p,times,twocolumn,authoryear]{elsarticle}

\usepackage{amssymb}
\usepackage{amsmath}
\usepackage{siunitx}
\usepackage{graphicx}
\usepackage{caption}
\setcitestyle{square}


\makeatletter
\def\ps@pprintTitle{%
  \let\@oddhead\@empty
  \let\@evenhead\@empty
  \let\@oddfoot\@empty
  \let\@evenfoot\@oddfoot
}
\makeatother

\begin{document}

\begin{frontmatter}

\title{Movement of charged particles in Earth's magnetic field and their relation to the Aurora lights}

\author[first]{Emil Spasov}
\affiliation[first]{organization={Norwegian University of Science and Technology},
%Department and Organization
            city={Trondheim},
            postcode={7491}, 
            country={Norway}}

\begin{abstract}
The current in Earth's outer core generates a magnetic field which can be approximated as the field of a magnetic dipole embedded in the center of the planet. The field interacts with particles in the solar winds and traps them in the so called van Allen belts. Using Euler's and Heun's methods for numerical approximation, we model these interactions and show the trajectories of positive charges (protons) inside the magnetic field. Notice that the particles come closest to Earth at the magnetic poles which explains why the Aurora lights are visible closest to the poles. Euler's method is shown to be inaccurate with error of up to 3000\% due to its weakness in dealing with rapid changes and accelerations.
\end{abstract}

\end{frontmatter}
\tableofcontents

\section{Introduction}
\label{introduction}
Earth's outer core is mostly composed of liquid iron and nickel. The motion of convection currents of the mixture generates a magnetic field that extends to space where it interacts with the solar winds. A solar wind is a stream of charged particles released from the upper atmosphere of the Sun. Some of the particles get trapped in Earths magnetic field in the so called van Allen Belts where they occasionally collide with atmospheric particles releasing tiny flashes of light that we see as Northern lights. By using numerical methods we simulate the movement of individual particles and study the phenomenon.

\section{Theory}
\label{theory}
We assume Earth's magnetic field to be that of a pure magnetic dipole due to the radius of Earth's core being much smaller than the distances we are analyzing.
The general expression for a magnetic dipole field \cite{textbook} is given by the following equation:
\begin{equation} \label{eq:B}
    \\ \vec{B} = \frac{\mu_0}{4\pi} \frac{3(\vec{m}\cdot\hat{r})\hat{r} - \vec{m}}{r^3}
\end{equation}
(where $r = \sqrt{x^2+y^2+z^2}$ is the distance from the dipole)
To find the magnetic dipole moment $\vec{m}$ we consider a coordinate system such that $\vec{m}=m\hat{j}$ is along the magnetic dipole. In this case $\vec{m}$ can be calculated using the experimentally determined value of the geomagnetic field at the equator $B_0 = 3.12 \cdot 10^{-5} \si{T}$ \cite{website}. The field at the equator is given by:
\begin{equation}\label{eq:B0}
    \\ \vec{B_0} = -\frac{\mu_0}{4\pi} \frac{m}{R_E^3} \hat{j}
\end{equation}
where $R_E = 6.378 \si{Mm}$ is Earth's radius at the equator.
By solving for $m$ and substituting in equation (\ref{eq:B}) we get the following expression for the magnetic dipole field of Earth:
\begin{equation} \label{eq:Bf}
    \\ \vec{B} = B_0 R_E^3 \frac{3(\hat{m}\cdot\hat{r})\hat{r} - \hat{m}}{r^3}
\end{equation}

We choose a coordinate system such that the x axis is from the Sun toward the Earth and
the z axis perpendicular to the ecliptic. The magnetic dipole is tilted by an angle $\alpha = 9.6^\circ$ with respect to the ecliptic. Because of that the $\vec{m}$ in Cartesian coordinates is:
\begin{gather}\label{eqs:m}
    m_x = m \sin{\alpha}
    \\
    m_y = 0
    \\
    m_z = m \cos{\alpha}
\end{gather} 
In this coordinate system the components of the magnetic field are:
\begin{gather}\label{eqs:B}
    B_x = B_0\frac{R_E^3}{r^3}\left(\frac{3( x m_x + z m_z)x}{r^2}-m_x\right)\\
    B_y = B_0\frac{R_E^3}{r^3}\left(\frac{3( x m_x + z m_z)y}{r^2}\right)\\
    B_z = B_0\frac{R_E^3}{r^3}\left(\frac{3( x m_x + z m_z)z}{r^2}-m_z\right)
\end{gather} 

The force of a magnetic field $\vec{B}$ on a moving charge $q$ with velocity $\vec{v}$ is given by the expression \cite{textbook}:
\begin{equation} \label{eqn:F}
    \vec{F} = q\vec{v}\times\vec{B}
\end{equation}
Using Newton's second law $F = ma$ we find the acceleration of the particle with mass $m$:
\begin{equation} \label{eqn:a}
    \vec{a} = \frac{q\vec{v}\times\vec{B}}{m}
\end{equation}
From (\ref{eqn:F}) we see that the magnetic force is perpendicular to the magnetic field and velocity of the particle. That means the work, defined as $W = \vec{F} \cdot \vec{v}t$, done by the magnetic field is always 0. From the expression for work $W = \Delta E_K = \frac{1}{2}m(v_1^2 - v_0^2) = 0$ we see that $v_0 = v_1$. In other words the velocity changes only in direction and not magnitude. 
\begin{equation} \label{eqn:dv}
    \Delta v = 0
\end{equation}

\section{Methods}
\label{methods}
To track the trajectory of particles we need to solve differential equations for motion over time with a non - constant acceleration. That can be done numerically using different methods for numerical approximation.

In this project we are using Python for the implementation of the numerical methods. To make our computations more efficient, we are using the Numpy library. The plots and graphs are created using standard functions from the Matplotlib library.

Two different methods for numerical approximation are used - Euler's and Heun's.

Euler's method is a first order Runge-Kutta method. It is a technique used to analyze a Differential Equation, which uses the idea of local linearity or linear approximation, where we use small tangent lines over a short distance to approximate the solution to an initial-value problem. Euler's method \cite{lecture} is generally defined as:
\begin{equation}\label{eqn:euler}
        y_n = y_{n-1} + h f(y_{n-1},t_{n-1})
\end{equation}
where $f=y'(y(t),t)$ is the rate of change of $y$ and $h$ is the step size.

Heun's method is pretty similar to Euler's, in fact it can be viewed as an extension of Euler's method into two-stage second-order Runge-Kutta method.
With this technique Euler's Method is used to roughly estimate the coordinates of the next point in the solution, and with this knowledge, the original estimate is re-predicted or corrected. Heun's method \cite{lecture} is defined as:
\begin{gather}\label{eqs:heun}
    \Tilde{y}_n = y_{n-1} + h f(y_{n-1},t_{n-1})
    \\
    y_n = y_{n-1} + \frac{h}{2} (f(y_{n-1},t_{n-1})+f(\Tilde{y}_n,t_n))
\end{gather} 
where as earlier $f$ is the rate of change and $\Tilde{y}$ is an approximation of the next step using Euler's method.

The numerical error using methods from the Runge-Kutta class accumulates with each step and is proportional to the step-size \cite{lecture} as shown in the equation:
\begin{equation}\label{eqn:error}
        ||e_N|| = Ch^p
\end{equation}
(where $e_N$ is the error in the N-th step, C is constant, h is the step-size and p is the order of the method under consideration) 

Euler's method is a first-order method but is relatively accurate over small intervals with small increments and as long as our function does not change too rapidly (too big $y'$). Big changes over short time create long tangent lines following which leads to large deviations. Consequently, we need to ensure that our step-size $h$ isn't too large or our numerical solution will be inaccurate.

Heun's method on the other hand is second-order method and is therefore significantly better, as we see from (\ref{eqn:error}) the error decreases exponentially with $p$.
\section{Results}
\label{Results}
\subsection{Earth's magnetic field}
The magnetic field is calculated using a 100x100 mesh grid. As mentioned earlier we are using coordinate system such that the x axis is from the Sun toward the Earth and the z axis perpendicular to the
ecliptic.

\begin{center} \label{fig: mfieldx}
\includegraphics[width=\columnwidth]{mfieldxz.pdf}
\captionof{figure}{The blue circle represent the Earth with the arrow showing the direction of the magnetic dipole moment. The red dashed lines represent the x and z axis. The stream lines show the direction of the magnetic field and their color represent the strength of the field as described in the color bar. The distances are shown in Mm (mega meter or $10^6\si{m}$). We notice that the strength of the field decreases rapidly as we go further away from the center, due to the factor $1/r^3$.}
\end{center}

\begin{center} \label{fig: mfieldxy}
\includegraphics[width=\columnwidth]{mfieldxy.pdf}
\captionof{figure}{Here again the blue circle represent the Earth with the streamline representing the direction and their color - the strength of the magnetic field. The red dashed lines represent the x and y axis. If the magnetic dipole was aligned with the z axis the streamlines would all point radially but due to the tilt and the x component of the magnetic dipole moment the magnetic field curls around the x axis (where y is relatively small and $xm_x$ dominates)}
\end{center}

\subsection{Particle trajectories}
We model the solar winds as positively charged particles (protons) with mass $m_p =  1.65 \times 10^{-27} \si{kg} $ and charge $q = +1.6 \times 10^{-19}\si{C}$. Particles are assumed to be point particles. To study different initial conditions we define 3 particles with the following initial position and velocity:

\begin{table}[h]
\centering
\begin{tabular}{c|ccc}
    & Particle 1 & Particle 2 & Particle 3\\ \cline{1-4}
$x_0$ & $-5R_E$      & $-4R_E$       & $-4.5R_E$    \\
$y_0$ & $0$          & $R_E$        & $R_E$       \\
$z_0$ & $R_E$        & $R_E$       & $0$        
\end{tabular}
\captionof{table}{Table with initial values for particle positions}
\end{table}

The usual speed of solar wind particles is about $24\times10^6 \si{m/min}$ therefore the magnitude of the velocity is set to be constant $24 \si{Mm/min}$ and the only thing changing is the direction. The angle between the initial trajectories and the x-axis are as follows: 
\begin{table}[h]
\centering
\begin{tabular}{c|ccc}
    & Particle 1 & Particle 2 & Particle 3\\ \cline{1-4}
$\theta$ & $\pi/8$      & $0$       & $\pi/7$    \\
$\phi$ & $0$          & $\pi/6$        & $\pi/4$       
\end{tabular}
\captionof{table}{Table with initial values for particle directions}
\end{table}

\begin{center} \label{fig: trajeuler}
\includegraphics[width=\columnwidth]{traj.pdf}
\captionof{figure}{Trajectories of 3 protons with Earth in the middle calculated using the simple Euler's method. The first row shows the particle trajectories as seen in the xy-plane with the x-axis in the horizontal and the y-axis in the vertical direction. In the same way the second row represents the same particles trajectories as seen in the xz-plane, with the z-axis in the vertical direction. All distances are represented in Mm (megameter/$10^6\si{m}$).}
\end{center}

We set the time step $\Delta t = 0.0001 \si{min}$ for an interval of 100 minutes.

\begin{center} \label{fig: trajheun}
\includegraphics[width=\columnwidth]{trajheun.pdf}
\captionof{figure}{The same 3 protons and initial conditions but this time calculated using Heun's method.} 
\end{center}

 We see particles in both figure 3 and figure 4 follow similar path. The trajectories in figure 4 however are more tightly packed, due to the better numerical approximation. The protons are following the field lines and orbiting around the center of the field. The trajectories in shown in both figures are as expected of charged particles in a magnetic dipole field.
 
 When particles from the solar winds come too close to Earth they collide with atmospheric particles and release tiny flashes of light that manifest as the famous Aurora lights. In both figures the particles come closest to Earth at the poles. That is the reason these phenomena are seen closest to Earth's magnetic poles.

\section{Discussion}
\label{discussion}
To check the accuracy of the results, we use (\ref{eqn:dv}) for both methods. The relative error is then given by $e_n = \frac{\Delta v_n}{v_0} =\frac{v_n - v_0}{v_0}$, where $v_n$ is the speed at the n-th step and $v_0$ is the initial speed.
\begin{center} \label{fig: err}
\includegraphics[width=\columnwidth]{err.pdf}
\captionof{figure}{Shown in the figure is the relative error $e_n$ in percent as a function of the step. The error given by Euler's method on the first row and by Heun's method on the second. We notice sudden raise of the error followed by a flat line with little to no change. This is caused by sudden change in the velocity and position due to the charge being in an area with a strong magnetic field. The charge then moves far from the source where the field is weak and there is little change in the velocity.}
\end{center}

As expected the error accumulates with each step with both methods. We notice that the error for Heun's method is smaller with a factor of $10^{-5}$. With a step-size of $h=10^{-4}$ this is in agreement with (\ref{eqn:error}) which states that Heun's method error must be at least $h$ times smaller than Euler's. 

We notice that in figure 3 the particles "fly out" of the graph after they get too close to the source of the magnetic field. This happens because Euler's method is very weak handling large and rapid changes in the function. In this case when the particle gets close to the center of the field it is exposed to a large magnetic force and rapid acceleration. The large acceleration leads to large velocity and displacement which explains the particle "flying" far from the center where the magnetic field is weaker.

The sudden movement of the protons far from the field source is represented by a sudden jump in the numerical error as seen in figure \ref{fig: err}. The magnetic field is much weaker at large distances and therefore the acceleration due to magnetic force is negligible. This leads to little to none numerical error after the jump.

\section{Summary and conclusions}
\label{summary}
We approximate the geomagnetic field to that of a magnetic dipole embedded in the center of Earth. The field is symmetric around the magnetic dipole moment vector.

Using Euler's and Heun's method of numerical approximation we simulate the movement of charged point particles trapped in the field and see that they follow the magnetic field lines while simultaneously orbiting around the source of the magnetic field. This is in agreement with the expected behaviour of a charge in a dipole field.

We notice that the particles get closest to Earth at the magnetic poles where they are able to interact with atmospheric particles. These interactions result in the famous polar lights.

Comparing the results yielded by two methods we notice the particles follow similar trends but there is a difference that can attributed to numerical error. Using the test condition of zero work done by the magnetic field we estimate numerical error using Euler's method of up to 3000\% mostly due to rapid change in the velocity and position, while Heun's only gives an error of up to 0.03\%. 

\bibliographystyle{elsarticle-harv} 
\begin{thebibliography}{3}
\bibitem{textbook}
David Jeffrey Griffiths and Cambridge University Press (2018). Introduction to electrodynamics. Cambridge I Pozostałe: Cambridge University Press.
\bibitem{website}
Wikipedia Contributors (2019). Earth’s magnetic field. [online] Wikipedia. Available at: https://en.wikipedia.org/wiki/Earth\%27s\_magnetic\_field.
\bibitem{lecture} Anne Kværnø (2023). Lecture notes: Numerical solution of ordinary differential equations.


\end{thebibliography}

%% else use the following coding to input the bibitems directly in the
%% TeX file.

%%\begin{thebibliography}{00}

%% Text of bibliographic item

%%\bibitem[ ()]{}

%%\end{thebibliography}
\end{document}

